\chapter{流形上的张量}\label{chapter_tensorfield}
我们回到几何学, 这一章我们讨论上一章的代数工具在微分几何中的具体应用.
我们首先会定义流形$M$上的张量场, 并讨论张量场与向量场的张量积之间的关系.
其次我们会讨论流形上的Riemann度量, 从而向Riemann几何迈出微小的一步.
这一节中我们也会引出单位分解定理, 并讨论它的应用.
最后我们会讨论交错映射构成的张量场, 也就是微分形式, 我们会略微谈及它的应用.

\section{张量场}
\begin{defn}
    微分流形$M$上一点$p$的\textbf{余切空间}是这一点处切空间的对偶空间, 记为$T^*_pM:=(T_pM)^*$.
    $T^*_pM$中的元素称为\textbf{余向量}.
\end{defn}

命题~\ref{defn of dual}~说明了$T^*_pM$具有向量空间的结构, 且维数与$M$相同.
那么类似命题~\ref{diff_stru_TM}~可以证明对$p\in M$, $T^*_pM$的不交并是一个微分流形.
我们于是有定义
\begin{defn}
    设$M$是微分流形, $\displaystyle T^*M:=\coprod_{p\in M}T^*_pM$是一个$2n$维微分流形, 定义为$M$的\textbf{余切丛}.
\end{defn}

由于任意向量空间的张量积都是向量空间, 所以$\tensor^{(r,s)}(V)$也是向量空间, 于是进一步地可以定义
\begin{defn}
    设$M$是微分流形, 定义$\displaystyle T^{(r,s)}TM:=\coprod_{p\in M}\tensor^{(r,s)}(T_pM)$为$M$的$(r,s)$-张量丛.
\end{defn}

仿照向量场的定义, 我们如下定义张量场:
\begin{defn}
    流形$M$上的一个$(r,s)$-张量场$A$是一个满足$\pi\circ A=1_M$的光滑映射$A:M\to T^{(r,s)}TM$, 其中$\pi:T^{(r,s)}TM\to M$是自然投影映射.
\end{defn}

在有了全局的张量场的定义之后, 我们考虑张量场在局部, 也就是某个坐标卡下的表现.
首先我们考虑余切空间.
\begin{lem}
    设$n$维流形$M$上的函数$x^1,\cdots,x^n$定义如引理~\ref{lem_indep}~所述, 那么$x^1_*|_p,\cdots,x^n_*|_p$是
    $\displaystyle\left.\frac{\partial}{\partial x^1}\right|_p,\cdots,\left.\frac{\partial}{\partial x^n}\right|_p$的对偶基.
    我们在这里将$T_{x^i(p)}\mathbb{R}$与$\mathbb{R}$视为等同的.
\end{lem}
\begin{proof}
    对任意$f\in C^\infty(\mathbb{R})$有
    \begin{align*}
        x^i_*|_p\left(\left.\frac{\partial}{\partial x^j}\right|_p\right)(f)&=\left.\frac{\partial f\circ x^i}{\partial x^j}\right|_p\\
        &=\frac{\partial f\circ x^i\circ\varphi^{-1}}{\partial u^j}\\
        &=\frac{\partial f\circ\pi^i}{\partial u^j}\\
        &=\frac{\d f}{\d x}\delta(i,j)
    \end{align*}
    注意到$\d/\d x\in T_{x^i(p)}\mathbb{R}$等同于$1\in\mathbb{R}$, 所以上式说明了$x^1_*|_p,\cdots,x^n_*|_p$是
    $\displaystyle\left.\frac{\partial}{\partial x^1}\right|_p,\cdots,\left.\frac{\partial}{\partial x^n}\right|_p$的对偶基.
\end{proof}

\begin{prop}\label{local tensor field}
    设$A$是一个$(r,s)$-张量场 \rmparen{不一定是光滑的}, 那么$A$是光滑张量场当且仅当对任意坐标卡$(U,\varphi)$, $A$都具有
    \begin{equation}
        \sum_{\substack{i_1,\cdots,i_k\\j_1,\cdots,j_l}}A^{i_1\cdots i_k}_{j_1\cdots j_l}\frac{\partial}{\partial x^{i_1}}\otimes\cdots\otimes\frac{\partial}{\partial x^{i_k}}\otimes x^{j_1}_*\otimes\cdots\otimes x^{j_l}_*\label{eq_local_tensorfield}
    \end{equation}
    的形式, 其中$A^{i_1\cdots i_k}_{j_1\cdots j_l}\in C^\infty(M)$.
\end{prop}
\begin{proof}
    验证定义~\ref{smooth function 2}~即可.
    对于张量来说, 写出坐标卡的具体表达式太过于复杂, 我们把这个工作留给读者.
\end{proof}

考虑一个共变张量丛$T^{(0,s)}TM$, 我们猜测它可以由$s$个余向量场 (即$1$阶共变张量场) 在$C^\infty(M)$上做张量得到.
同时我们也很希望余向量场是向量场的对偶, 即$T^{(0,1)}TM=\Hom_{C^\infty(M)}(TM,\mathbb{R})$.
依照这种想法, 我们有如下的结论:

\begin{prop}\label{field and map}
    流形$M$上的任一$s$阶共变张量场$A$均对应了一个逐点定义的$k$重$C^\infty(M)$线性映射$a:\underbrace{\mfrakX(M)\times\cdots\mfrakX(M)}_{k\text{个}}\to C^\infty(M)$:
    对任意$p\in M$, 取一个含$p$的坐标卡$(U,\varphi)$使得$A$具有~\eqref{eq_local_tensorfield}~的形式, 定义
    \begin{align*}
        a|_U&=\sum_{i_1,\cdots,i_k}A_{i_1\cdots i_k}x^{i_1}_*\otimes\cdots\otimes x^{i_k}_*\\
        (X_1|_p,\cdots,X_k|_p)&\mapsto\sum_{i_1,\cdots,i_k}A_{i_1\cdots i_k}(p)x^{i_1}_*(X_1|_p)\cdots x^{i_k}_*(X_k|_p)
    \end{align*}
    进一步地, 这个对应是一一的.
\end{prop}

比较命题~\ref{field and map}~与开始的想法可以发现, 我们仅仅使用了最传统的多重线性映射的语言.
这是因为第\ref{chapter_tensor}章中的张量与多重线性映射之间的一系列命题只在向量空间上成立, 而我们现在在处理环上的模, 因此我们就只能用多重线性映射来说这件事.

我们需要以下一些引理.

\begin{lem}\label{local operator}
    设$f:\mfrakX(M)\to C^\infty(M)$是$C^\infty(M)$-线性的, 那么如果向量场$X$在某个开集$U$上恒为$0$, 则$f(X)$在$U$上也恒取$0$.
\end{lem}
\begin{proof}
    设$p\in U$, 取$p$在$U$中的一个邻域$V$及一个冲击函数$h$, 使得$h$在$V$上恒取$1$, $\supp{h}\subset U$.
    那么在$M$上$hX=0$, 从而有$f(hX)=0$.
    而由$f$的$C^\infty(M)$-线性性, 有$hf(X)=0$, 在$V$上就有$f(X)=0$.
    因此$f(X)(p)=0$, 由$p$的任意性可知$f(X)$在$U$上恒为$0$.
\end{proof}

\begin{lem}\label{point operator}
    设$f:\mfrakX(M)\to C^\infty(M)$是$C^\infty(M)$-线性的, 则向量场$X$在某点$p$处为$0$时也有$f(X)(p)=0$.
\end{lem}
\begin{proof}
    取一个包含$p$的坐标卡$(U,\varphi)$, 我们定义$f$在$U$上的\textit{限制}.
    对一个向量场$Y\in\mfrakX(U)$, 按命题~\ref{vector field extension}~取一个向量场$\widetilde{Y}\in\mfrakX(M)$使得$\widetilde{Y}|_U=Y$,
    \footnote{严格来说, \ref{vector field extension}~无法保证$Y$与$\widetilde{Y}$在整个$U$上相等, 我们需要利用流形的\textit{正规性} (\Parencite[定理4.81]{Lee_IntroTopoMani}) 来取一个更大的开集$V$包含$U$, 然后构造冲击函数完成证明.}
    定义$f|_U(Y)=f(\widetilde{Y})$.
    $f|_U$是良定义的, 因为引理~\ref{local operator}~保证了如果$Y_1|_U=Y_2|_U$, 那么一定有$f|_U(Y_1|_U)=f|_U(Y_2|_U)$.
    由于$U$上存在局部自然标架场, 所以$\mfrakX(U)$是自由$C^\infty(M)$-模 (请回忆相关定义),
    因此$f|_U$的作用可以被矩阵表示.
    设$f|_U$的矩阵为$A$, 对任意一个$X|_U$, 设其坐标为$\alpha=(\alpha^1,\cdots,\alpha^n)^\top$.
    那么有
    \begin{align}
        f(X)(p)&=f|_U(X|_U)(p)\\
        &=A(\alpha^1(p),\cdots,\alpha^n(p))\\
        &=A(0,\cdots,0)\\
        &=0\qedhere
    \end{align}
\end{proof}

\begin{col}
    显然引理~\ref{local operator}~与引理~\ref{point operator}~均对多线性映射成立.
\end{col}

我们着手证明命题~\ref{field and map}.
\begin{proof}[{\bf 命题~\ref{field and map}~的证明}]
    显然这个对应是单射, 我们只需要证明它是满射.
    设$a:\underbrace{\mfrakX(M)\times\cdots\mfrakX(M)}_{k\text{个}}\to C^\infty(M)$是$k$重线性映射.
    在每一点$p$处我们定义一个$k$重线性映射$B(p):\underbrace{T_pM\times\cdots T_pM}_{k\text{个}}\to\mathbb{R}$如下:
    对$v_1,v_2,\cdots,v_k\in T_pM$, 取向量场$X_1,\cdots,X_k$使得$X_i|_p=v_i,\ i=1,2,\cdots,k$ (存在性由向量场扩张引理保证),
    则定义$B(p)(v_1,\cdots,v_k)=a(X_1|_p,\cdots,X_k|_p)$.
    如果另有向量场$Y_1,\cdots,Y_k$满足$Y_i|_p=v_i,\ i=1,2,\cdots,k$, 那么考虑
    \begin{align}
        a(X_1,\cdots,X_k)-a(Y_1,\cdots,Y_k)&=\sum_{i=1}^k\left(a(X_1,\cdots,X_i,Y_{i+1},\cdots,Y_k)\right.\notag\\
        &\qquad\left.-a(X_1,\cdots,Y_i,Y_{i+1},\cdots,Y_k)\right)\notag\\
        &=\sum_{i=1}^ka(X_1,\cdots,X_i-Y_i,Y_{i+1},\cdots,Y_k)\label{eq_field and map}
    \end{align}
    注意到~\eqref{eq_field and map}~式的第$i$个分量在$p$处取$0$, 所以由引理~\ref{point operator}~知上式在$p$处取$0$, 即
    \[a(X_1|_p,\cdots,X_k|_p)=a(Y_1|_p,\cdots,Y_k|_p)\]
    从而$B(p)$是一个良定义的多重线性映射, 那么$B:M\to T^{(0,k)}TM$是一个$k$阶共变张量场.
\end{proof}

\section{Riemann度量}
第\ref{chapter_manifolds}章与上一节的内容属于流形的\textit{微分拓扑}, 而我们在这一节将会引入我们在第~\ref{metric_first}~节提到的属于\textit{微分几何}的一个结构: \textit{Riemann度量}.

Riemann度量是欧氏空间中内积的推广, Riemann度量让我们能够在流形上定义长度, 角度, 面积等等一系列几何量, 并给出较之前几章多得多的几何信息.
同时, Riemann度量也使得\textit{等距变换群}有了意义, 从而我们进入了Erlangen纲领的轨道上, 正式开始了``几何学''的讨论.
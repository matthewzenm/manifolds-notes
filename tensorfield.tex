\chapter{流形上的张量}\label{chapter_tensorfield}
我们回到几何学, 这一章我们讨论上一章的代数工具在微分几何中的具体应用.
我们首先会定义流形$M$上的张量场, 并讨论张量场与向量场的张量积之间的关系.
其次我们会讨论交错映射构成的张量场, 也就是微分形式, 我们会略微谈及它的应用.
最后我们会讨论流形上的Riemann度量, 从而向Riemann几何迈出微小的一步.

\section{张量场}\label{section_tensorfield}
\begin{defn}
    微分流形$M$上一点$p$的\textbf{余切空间}是这一点处切空间的对偶空间, 记为$T^*_pM:=(T_pM)^*$.
    $T^*_pM$中的元素称为\textbf{余向量}.
\end{defn}

命题~\ref{defn of dual}~说明了$T^*_pM$具有向量空间的结构, 且维数与$M$相同.
那么类似命题~\ref{diff_stru_TM}~可以证明对$p\in M$, $T^*_pM$的不交并是一个微分流形.
我们于是有定义
\begin{defn}
    设$M$是微分流形, $\displaystyle T^*M:=\coprod_{p\in M}T^*_pM$是一个$2n$维微分流形, 定义为$M$的\textbf{余切丛}.
\end{defn}

由于任意向量空间的张量积都是向量空间, 所以$\tensor^{(r,s)}(V)$也是向量空间, 于是进一步地可以定义
\begin{defn}
    设$M$是微分流形, 定义$\displaystyle T^{(r,s)}TM:=\coprod_{p\in M}\tensor^{(r,s)}(T_pM)$为$M$的$(r,s)$-张量丛.
\end{defn}

仿照向量场的定义, 我们如下定义张量场:
\begin{defn}
    流形$M$上的一个$(r,s)$-张量场$A$是一个满足$\pi\circ A=1_M$的光滑映射$A:M\to T^{(r,s)}TM$, 其中$\pi:T^{(r,s)}TM\to M$是自然投影映射.
\end{defn}

在有了全局的张量场的定义之后, 我们考虑张量场在局部, 也就是某个坐标卡下的表现.
首先我们考虑余切空间.
\begin{lem}
    设$n$维流形$M$上的函数$x^1,\cdots,x^n$定义如引理~\ref{lem_indep}~所述, 那么$x^1_*|_p,\cdots,x^n_*|_p$是
    $\displaystyle\left.\frac{\partial}{\partial x^1}\right|_p,\cdots,\left.\frac{\partial}{\partial x^n}\right|_p$的对偶基.
    我们在这里将$T_{x^i(p)}\mathbb{R}$与$\mathbb{R}$视为等同的.
\end{lem}
\begin{proof}
    对任意$f\in C^\infty(\mathbb{R})$有
    \begin{align*}
        x^i_*|_p\left(\left.\frac{\partial}{\partial x^j}\right|_p\right)(f)&=\left.\frac{\partial f\circ x^i}{\partial x^j}\right|_p\\
        &=\frac{\partial f\circ x^i\circ\varphi^{-1}}{\partial u^j}\\
        &=\frac{\partial f\circ\pi^i}{\partial u^j}\\
        &=\frac{\d f}{\d x}\delta(i,j)
    \end{align*}
    注意到$\d/\d x\in T_{x^i(p)}\mathbb{R}$等同于$1\in\mathbb{R}$, 所以上式说明了$x^1_*|_p,\cdots,x^n_*|_p$是
    $\displaystyle\left.\frac{\partial}{\partial x^1}\right|_p,\cdots,\left.\frac{\partial}{\partial x^n}\right|_p$的对偶基.
\end{proof}

\begin{prop}\label{local tensor field}
    设$A$是一个$(r,s)$-张量场 \rmparen{不一定是光滑的}, 那么$A$是光滑张量场当且仅当对任意坐标卡$(U,\varphi)$, $A$都具有
    \begin{equation}
        \sum_{\substack{i_1,\cdots,i_k\\j_1,\cdots,j_l}}A^{i_1\cdots i_k}_{j_1\cdots j_l}\frac{\partial}{\partial x^{i_1}}\otimes\cdots\otimes\frac{\partial}{\partial x^{i_k}}\otimes x^{j_1}_*\otimes\cdots\otimes x^{j_l}_*\label{eq_local_tensorfield}
    \end{equation}
    的形式, 其中$A^{i_1\cdots i_k}_{j_1\cdots j_l}\in C^\infty(M)$.
\end{prop}
\begin{proof}
    验证定义~\ref{smooth function 2}~即可.
    对于张量来说, 写出坐标卡的具体表达式太过于复杂, 我们把这个工作留给读者.
\end{proof}

考虑一个共变张量丛$T^{(0,s)}TM$, 我们猜测它可以由$s$个余向量场 (即$1$阶共变张量场) 在$C^\infty(M)$上做张量得到.
同时我们也很希望余向量场是向量场的对偶, 即$T^{(0,1)}TM=\Hom_{C^\infty(M)}(TM,\mathbb{R})$.
依照这种想法, 我们有如下的结论:

\begin{prop}\label{field and map}
    流形$M$上的任一$s$阶共变张量场$A$均对应了一个逐点定义的$k$重$C^\infty(M)$线性映射$a:\underbrace{\mathfrak{X}(M)\times\cdots\mathfrak{X}(M)}_{k\text{个}}\to C^\infty(M)$:
    对任意$p\in M$, 取一个含$p$的坐标卡$(U,\varphi)$使得$A$具有~\eqref{eq_local_tensorfield}~的形式, 定义
    \begin{align*}
        a|_U&=\sum_{i_1,\cdots,i_k}A_{i_1\cdots i_k}x^{i_1}_*\otimes\cdots\otimes x^{i_k}_*\\
        (X_1|_p,\cdots,X_k|_p)&\mapsto\sum_{i_1,\cdots,i_k}A_{i_1\cdots i_k}(p)x^{i_1}_*(X_1|_p)\cdots x^{i_k}_*(X_k|_p)
    \end{align*}
    进一步地, 这个对应是一一的.
\end{prop}

比较命题~\ref{field and map}~与开始的想法可以发现, 我们仅仅使用了最传统的多重线性映射的语言.
这是因为第\ref{chapter_tensor}章中的张量与多重线性映射之间的一系列命题只在向量空间上成立, 而我们现在在处理环上的模, 因此我们就只能用多重线性映射来说这件事.

我们需要以下一些引理.

\begin{lem}\label{local operator}
    设$f:\mathfrak{X}(M)\to C^\infty(M)$是$C^\infty(M)$-线性的, 那么如果向量场$X$在某个开集$U$上恒为$0$, 则$f(X)$在$U$上也恒取$0$.
\end{lem}
\begin{proof}
    设$p\in U$, 取$p$在$U$中的一个邻域$V$及一个冲击函数$h$, 使得$h$在$V$上恒取$1$, $\supp{h}\subset U$.
    那么在$M$上$hX=0$, 从而有$f(hX)=0$.
    而由$f$的$C^\infty(M)$-线性性, 有$hf(X)=0$, 在$V$上就有$f(X)=0$.
    因此$f(X)(p)=0$, 由$p$的任意性可知$f(X)$在$U$上恒为$0$.
\end{proof}

\begin{lem}\label{point operator}
    设$f:\mathfrak{X}(M)\to C^\infty(M)$是$C^\infty(M)$-线性的, 则向量场$X$在某点$p$处为$0$时也有$f(X)(p)=0$.
\end{lem}
\begin{proof}
    取一个包含$p$的坐标卡$(U,\varphi)$, 我们定义$f$在$U$上的\textit{限制}.
    对一个向量场$Y\in\mathfrak{X}(U)$, 按命题~\ref{vector field extension}~取一个向量场$\widetilde{Y}\in\mathfrak{X}(M)$使得$\widetilde{Y}|_U=Y$,
    \footnote{严格来说, \ref{vector field extension}~无法保证$Y$与$\widetilde{Y}$在整个$U$上相等, 我们需要利用流形的\textit{正规性} (\Parencite[定理4.81]{Lee_IntroTopoMani}) 来取一个更大的开集$V$包含$U$, 然后构造冲击函数完成证明.}
    定义$f|_U(Y)=f(\widetilde{Y})$.
    $f|_U$是良定义的, 因为引理~\ref{local operator}~保证了如果$Y_1|_U=Y_2|_U$, 那么一定有$f|_U(Y_1|_U)=f|_U(Y_2|_U)$.
    由于$U$上存在局部自然标架场, 所以$\mathfrak{X}(U)$是自由$C^\infty(M)$-模 (请回忆相关定义),
    因此$f|_U$的作用可以被矩阵表示.
    设$f|_U$的矩阵为$A$, 对任意一个$X|_U$, 设其坐标为$\alpha=(\alpha^1,\cdots,\alpha^n)^\top$.
    那么有
    \begin{align*}
        f(X)(p)&=f|_U(X|_U)(p)\\
        &=A(\alpha^1(p),\cdots,\alpha^n(p))\\
        &=A(0,\cdots,0)\\
        &=0\qedhere
    \end{align*}
\end{proof}

\begin{col}
    显然引理~\ref{local operator}~与引理~\ref{point operator}~均对多线性映射成立.
\end{col}

我们着手证明命题~\ref{field and map}.
\begin{proof}[{\bf 命题~\ref{field and map}~的证明}]
    显然这个对应是单射, 我们只需要证明它是满射.
    设$a:\underbrace{\mathfrak{X}(M)\times\cdots\mathfrak{X}(M)}_{k\text{个}}\to C^\infty(M)$是$k$重线性映射.
    在每一点$p$处我们定义一个$k$重线性映射$B(p):\underbrace{T_pM\times\cdots T_pM}_{k\text{个}}\to\mathbb{R}$如下:
    对$v_1,v_2,\cdots,v_k\in T_pM$, 取向量场$X_1,\cdots,X_k$使得$X_i|_p=v_i,\ i=1,2,\cdots,k$ (存在性由向量场扩张引理保证),
    则定义$B(p)(v_1,\cdots,v_k)=a(X_1|_p,\cdots,X_k|_p)$.
    如果另有向量场$Y_1,\cdots,Y_k$满足$Y_i|_p=v_i,\ i=1,2,\cdots,k$, 那么考虑
    \begin{align}
        a(X_1,\cdots,X_k)-a(Y_1,\cdots,Y_k)&=\sum_{i=1}^k\left(a(X_1,\cdots,X_i,Y_{i+1},\cdots,Y_k)\right.\notag\\
        &\qquad\left.-a(X_1,\cdots,Y_i,Y_{i+1},\cdots,Y_k)\right)\notag\\
        &=\sum_{i=1}^ka(X_1,\cdots,X_i-Y_i,Y_{i+1},\cdots,Y_k)\label{eq_field and map}
    \end{align}
    注意到~\eqref{eq_field and map}~式的第$i$个分量在$p$处取$0$, 所以由引理~\ref{point operator}~知上式在$p$处取$0$, 即
    \[a(X_1|_p,\cdots,X_k|_p)=a(Y_1|_p,\cdots,Y_k|_p)\]
    从而$B(p)$是一个良定义的多重线性映射, 那么$B:M\to T^{(0,k)}TM$是一个$k$阶共变张量场.
\end{proof}

\section{微分形式}\label{section_forms}
\begin{defn}
    流形$M$上的一个\textbf{微分$k$-形式}是指$M$上的一个交错的$k$-阶共变张量场, 即一个光滑映射
    \[\alpha:M\to\coprod_{p\in M}\left(\bigwedge\nolimits^kT_pM\right)\]
    且满足$\pi\circ\alpha=1_M$.
    微分$k$-形式也简称\textbf{$k$-形式}.
\end{defn}

\begin{sym}
    我们把$M$上所有微分$k$-形式的集合记为$\form^k(M)$.
    按照逐点定义的加法与数乘, $\form^k(M)$构成一个$C^\infty(M)$模, 这是我们在向量场处就熟知的.
    与向量场不同的是, 微分形式之间还可以逐点定义楔积$\form^k(M)\times\form^l(m)\to\form^{k+1}(M)$, 这使得$\displaystyle C^\infty(M)\oplus\left(\bigoplus_{k\geq 1}\form^k(M)\right)$成为一个分次代数.
\end{sym}

我们在定义~\ref{dual map}~处定义了对偶空间之间的对偶映射, 而微分形式生活在余切空间 (的楔积) 之中, 所以我们想要将对偶映射推广到微分形式上.

\begin{defn}\label{pullback_def}
    设$f:M\to N$是流形间的光滑映射, 那么定义$f^*:\form^k(N)\to\form^k(M)$将$\omega$映为$f^*\omega$, 且$f^*\omega$满足
    \begin{align*}
        (f^*\omega)(v_1,\cdots,v_k)&=\omega(f_*(v_1),\cdots,f_*(v_k))
    \end{align*}
    $f^*$称为由$f$诱导的$N$到$M$的\textbf{拉回映射}.
    此外, 为简便起见, 所有由$f$诱导的$\form^n(N)\to\form^n(M),\ n\in\mathbb{N}$的拉回映射都会使用同一个记号$f^*$.
\end{defn}

把向量空间看成最简单的流形, 那么定义~\ref{dual map}~可以看作是拉回映射的特例.

\begin{prop}[拉回映射的性质]
    设$M,N,P$是微分流形, 光滑映射$M\xrightarrow{f}N\xrightarrow{g}P$, 那么有以下命题成立:
    \begin{enumerate}[{\rm 1.}]
        \item \rmparen{$\mathbb{R}$-线性性}\ $f^*:\form^n(N)\to\form^n(M),\ n\in\mathbb{N}$都是$\mathbb{R}$-线性的;
        \item \rmparen{函子性}\ 对$\omega\in\form^k(P)$, 有
        \[(f\circ g)^*\omega=g^*(f^*\omega)\]
        \item 设$h\in C^\infty(N)$, 那么有
        \[f^*(h\omega)=(h\circ f)f^*\omega\]
        \item 对形式$\omega,\eta$, 有
        \[f^*(\omega\wedge\eta)=f^*\omega\wedge f^*\eta\]
    \end{enumerate}
\end{prop}
\begin{proof}
    简单计算.
\end{proof}

\begin{eg}\label{pullback of top}
    设$M,N$都是$n$维流形, $f:M\to N$是光滑映射.
    对一个$\omega\in\form^n(N)$, 设$\omega$在$N$的某个坐标卡$(V,\psi)$具有形式$hy^1_*\wedge\cdots\wedge y^n_*$, 其中$h\in C^\infty(N)$ (回忆命题~\ref{basis of exterior}).
    我们考虑$\omega$的拉回.
    在$M$对应的坐标卡$(U,\varphi)$下计算有
    \begin{align*}
        f^*\omega&=f^*(hy^1_*\wedge\cdots\wedge y^n_*)\\
        &=(h\circ f)(y^1\circ f)_*\wedge\cdots\wedge(y^n\circ f)_*
    \end{align*}
    设$\displaystyle(y^i\circ f)_*:=\sum_{j}a^i_jx^j_*$, 那么矩阵
    \[A=\begin{bmatrix}
        a^1_1 & a^1_2 & \cdots & a^1_n\\
        a^2_1 & a^2_2 & \cdots & a^2_n\\
        \vdots & \vdots & \ddots & \vdots\\
        a^n_1 & a^n_2 & \cdots & a^n_n
    \end{bmatrix}\]
    即为$f$在基$x^i_*$与$y^i_*$下的Jacobi矩阵, 从而有
    \begin{align*}
        f^*\omega&=(h\circ f)(y^1\circ f)_*\wedge\cdots\wedge(y^n\circ f)_*\\
        &=(h\circ f)\left(\sum_{j}a^1_jx^j_*\right)\wedge\cdots\wedge\left(\sum_{j}a^n_jx^j_*\right)\\
        &=(h\circ f)\det{A}x^1_*\wedge\cdots\wedge x^n_*\quad\text{(回忆命题~\ref{exitstence and uniqueness of determinant})}\\
        &=(h\circ f)\det(\d{f})x^1_*\wedge\cdots\wedge x^n_*
    \end{align*}
\end{eg}

微分形式最重要的一个运算是外微分运算.

\begin{defn}
    如果一族$\mathbb{R}$-线性映射$\d{}:\form^{k+1}(M)\to\form^k(M)$ (对每个$k\in\mathbb{N}$都用相同记号) 满足
    \begin{enumerate}
        \item $\d{}\circ\d=0$;
        \item 对$\omega\in\form^k(M),\eta\in\form^l(M)$有$\d(\omega\wedge\eta)=\d\omega\wedge\eta+(-1)^k\omega\wedge\d\eta$;
        \item 对任意$f\in C^\infty(M)$, $\d{f}$是$f$的微分 (张量场), 满足$\d{f}(X)=Xf$,
    \end{enumerate}
    那么称$\d$为$M$上的一个\textbf{外微分}.
\end{defn}

\begin{sym}
    由于$x^i_*=\d{x^i}$, 我们从现在开始弃用$x^i_*\ (i=1,2,\cdots,n)$表示余标架场的记号, 全部使用$\d{x^i}\ (i=1,2,\cdots,n)$.

    请时刻记住$\d{}$是一族映射.
\end{sym}

如同行列式, 我们也证明外微分是存在且唯一的.
\begin{prop}\label{exterior differential}
    每个流形$M$上存在唯一的一族外微分映射.
\end{prop}
\begin{proof}
    对$h\in C^\infty(M)$, 定义$\d{h}=h_*$.
    设$\omega\in\form^k(M)\ (k\geq 1)$, 取$M$的任一坐标卡$(U,\varphi)$, 设
    \[\omega=\sum_{i_1,\cdots,i_k}A_{i_1,\cdots,i_k}\d x^{i_1}\wedge\cdots\wedge\d x^{i_k}\]
    定义
    \[\d\omega=\sum_{i_1,\cdots,i_k}\d{A_{i_1,\cdots,i_k}}\wedge\d x^{i_1}\wedge\cdots\wedge\d x^{i_k}\]
    我们证明这个定义是相容的.
    假设另一个坐标卡$(V,\psi)$与$U$交非空, 且在$(V,\psi)$上
    \[\omega=\sum_{i_1,\cdots,i_k}B_{i_1,\cdots,i_k}\d y^{i_1}\wedge\cdots\wedge\d y^{i_k}\]
    简记$\d x^{i_1,\cdots,i_k}:=\d x^{i_1}\wedge\cdots\wedge\d x^{i_k},\d y^{i_1,\cdots,i_k}:=\d y^{i_1}\wedge\cdots\wedge\d y^{i_k}$.
    那么由相等可知
    \begin{align*}
        \varphi^*&\left(\sum_{i_1,\cdots,i_k}A_{i_1,\cdots,i_k}\d x^{i_1,\cdots,i_k}\right)=
        \psi^*\left(\sum_{i_1,\cdots,i_k}B_{i_1,\cdots,i_k}\d y^{i_1,\cdots,i_k}\right)\\
        \implies&\begin{array}{>{\displaystyle}r>{\displaystyle}l}\displaystyle
            &\sum_{i_1,\cdots,i_k}A_{i_1,\cdots,i_k}\d x^{i_1,\cdots,i_k}\\
            =&\sum_{i_1,\cdots,i_k}(\varphi^{-1}\circ\psi)^*B_{i_1,\cdots,i_k}(\varphi^{-1}\circ\psi)^*(\d y^{i_1,\cdots,i_k})
        \end{array}\\
        \implies&\begin{array}{>{\displaystyle}r>{\displaystyle}l}\displaystyle
            &\sum_{i_1,\cdots,i_k}A_{i_1,\cdots,i_k}\d x^{i_1,\cdots,i_k}\\
            =&\sum_{i_1,\cdots,i_k}\det{\d(\varphi^{-1}\circ\psi)}(B_{i_1,\cdots,i_k}\circ\psi\circ\varphi^{-1})\d x^{i_1,\cdots,i_k}
        \end{array}
    \end{align*}
    最后一行是例~\ref{pullback of top}~的结果.
    那么计算两个坐标卡上的外微分有
    \[\begin{array}{>{\displaystyle}c>{\displaystyle}r>{\displaystyle}l}
        &&\sum_{i_1,\cdots,i_k}\d{A_{i_1,\cdots,i_k}}\wedge \d x^{i_1,\cdots,i_k}\\
        &=&\sum_{i_1,\cdots,i_k}\d\left(\det{\d(\varphi^{-1}\circ\psi)}B_{i_1,\cdots,i_k}\circ\psi\circ\varphi^{-1}\right)\wedge \d x^{i_1,\cdots,i_k}\\
        &=&\sum_{i_1,\cdots,i_k}\det{\d(\varphi^{-1}\circ\psi)}\d{B_{i_1,\cdots,i_k}}\circ\psi_*\circ\varphi^{-1}_*\wedge \d x^{i_1,\cdots,i_k}\\
        &=&\sum_{i_1,\cdots,i_k}(\varphi^{-1}\circ\psi)^*(\d{B_{i_1,\cdots,i_k}})\wedge(\varphi^{-1}\circ\psi)^*(\d y^{i_1,\cdots,i_k})\\
        \implies&&(\varphi)^*\left(\sum_{i_1,\cdots,i_k}\d{A_{i_1,\cdots,i_k}}\wedge \d x^{i_1,\cdots,i_k}\right)\\
        &=&(\psi)^*\left(\sum_{i_1,\cdots,i_k}\d{B_{i_1,\cdots,i_k}}\d y^{i_1,\cdots,i_k}\right)
    \end{array}\]
    从而外微分是相容的.

    我们证明这个$\d{}$满足外微分的公理, 按我们的构造, 这只需要在一个坐标卡上考虑.
    显然$\d{}$是$\mathbb{R}$-线性的.
    我们验证$\d{}\circ\d=0$.
    容易注意到对$f\in C^\infty(M)$有
    \[\d{f}=\sum_{i}\frac{\partial f}{\partial x^i}\d{x^i}\]
    对一个$\displaystyle\omega=\sum_{i_1,\cdots,i_k}A_{i_1,\cdots,i_k}\d{x^{i_1}}\wedge\cdots\wedge\d{x^{i_k}}$, 有
    \begin{align*}
        \d(\d\omega)&=\d\left(\sum_{i_1,\cdots,i_k}\d{A_{i_1,\cdots,i_k}}\wedge\d{x^{i_1}}\wedge\cdots\wedge\d{x^{i_k}}\right)\\
        &=\d\left(\sum_{r,i_1,\cdots,i_k}\frac{\partial A_{i_1,\cdots,i_k}}{\partial x^r}\d{x^r}\wedge\d{x^{i_1}}\wedge\cdots\wedge\d{x^{i_k}}\right)\\
        &=\sum_{r,s,i_1,\cdots,i_k}\frac{\partial^2 A_{i_1,\cdots,i_k}}{\partial x^r\partial x^s}\wedge\d{x^s}\wedge\d{x^r}\wedge\d{x^{i_1}}\wedge\cdots\wedge\d{x^{i_k}}
    \end{align*}
    而注意到系数$\dfrac{\partial^2 A_{i_1,\cdots,i_k}}{\partial x^r\partial x^s}$同时对应了$\d{x^s}\wedge\d{x^r}\wedge\cdots$与$\d{x^r}\wedge\d{x^s}\wedge\cdots$两个形式, 所以互相抵消, 有$\d(\d\omega)=0$.
    其次对
    \begin{gather*}
        \omega=\sum_{i_1,\cdots,i_k}A_{i_1,\cdots,i_k}\d{x^{i_1,\cdots,i_k}}\\
        \eta=\sum_{j_1,\cdots,j_l}B_{j_1,\cdots,j_l}\d{x^{j_1,\cdots,j_l}}
    \end{gather*}
    有
    \begin{align*}
        \d(\omega\wedge\eta)&=\d\left(\sum_{\substack{i_1,\cdots,i_k\\j_1,\cdots,j_l}}A_{i_1,\cdots,i_k}B_{j_1,\cdots,j_l}\d{x^{i_1,\cdots,i_k}}\wedge\d{x^{j_1,\cdots,j_l}}\right)\\
        &\begin{array}{r>{\displaystyle}l}
            =& \sum_{\substack{i_1,\cdots,i_k\\j_1,\cdots,j_l}}(B_{j_1,\cdots,j_l}\d{A_{i_1,\cdots,i_k}}+A_{i_1,\cdots,i_k}\d{B_{j_1,\cdots,j_l}})\wedge\\
            &\wedge\d{x^{i_1,\cdots,i_k}}\wedge\d{x^{j_1,\cdots,j_l}}
        \end{array}\\
        &\begin{array}{r>{\displaystyle}l}
            =&\sum_{\substack{i_1,\cdots,i_k\\j_1,\cdots,j_l}}B_{j_1,\cdots,j_l}\d{A_{i_1,\cdots,i_k}}\wedge\d{x^{i_1,\cdots,i_k}}\wedge\d{x^{j_1,\cdots,j_l}}\\
            &+(-1)^k\sum_{\substack{i_1,\cdots,i_k\\j_1,\cdots,j_l}}A_{i_1,\cdots,i_k}\d{x^{i_1,\cdots,i_k}}\wedge\d{B_{j_1,\cdots,j_l}}\wedge\d{x^{j_1,\cdots,j_l}}
        \end{array}\\
        &=\d\omega\wedge\eta+(-1)^k\omega\wedge\d\eta
    \end{align*}
    其中第2个等号是Leibniz法则, 第3个等号是反交换律.
    所以$\d{}$满足$\d(\omega\wedge\eta)=\d\omega\wedge\eta+(-1)^k\omega\wedge\d\eta$.
    按定义$\d{}$一定满足$\d{f}(X)=Xf$.

    以下我们证明外微分的唯一性.
    只需要在一个坐标卡上证明任意外微分$\delta$都满足$\d{}$的定义, 我们对$k$使用归纳法.
    $k=1$时, 由外微分的定义一定有$\delta:C^\infty\to\form^1(M)$与$\d{}:C^\infty\to\form^1(M)$相等.
    那么$\delta{x^i}=\d{x^i}\ (i=1,2,\cdots,n)$余切向量场.
    由$\mathbb{R}$-线性性, 只需考虑单项式即可, 那么考虑$f\d{x^i}$, 有
    \begin{align*}
        \delta(f\d{x^i})&=\d{f}\wedge\d{x^i}-f\delta(\delta{x^i})\\
        &=\d{f}\wedge\d{x^i}
    \end{align*}
    所以$\delta:\form^1(M)\to\form^2(M)$与$\d{}:\form^1(M)\to\form^2(M)$相等.
    假设命题在$k$处成立, 即$\delta:\form^{k-1}(M)\to\form^k(M)$与$\d{}:\form^{k-1}(M)\to\form^k(M)$相等.
    考虑$\omega=f\d x^{i_1}\wedge\d x^{i_2}\wedge\cdots\wedge\d x^{i_{k+1}}$, 记$I$为取值为$1$的常值函数, 则
    \begin{align*}
        \d\omega&=\d((f\d x^{i_1})\wedge\d x^{i_2}\wedge\cdots\wedge\d x^{i_{k+1}})\\
        &=\d(f\d x^{i_1})\wedge\d x^{i_2}\wedge\cdots\wedge\d x^{i_{k+1}}+f\d x^{i_1}\wedge\d(I\d x^{i_2}\wedge\cdots\wedge\d x^{i_{k+1}})\\
        &=\d f\wedge\d x^{i_1}\wedge\d x^{i_2}\wedge\cdots\wedge\d x^{i_{k+1}}\quad(\d{I}=0)
    \end{align*}
    因此$\delta:\form^{k}(M)\to\form^{k+1}(M)$与$\d{}:\form^{k}(M)\to\form^{k+1}(M)$相等.
    从而由归纳法可知$\delta=\d{}$, 即外微分一定是唯一的.
\end{proof}

命题~\ref{exterior differential}~同时给出了计算外微分的方法.
此外, 在计算外微分时我们也会用到如下命题:

\begin{prop}[外微分的自然性]
    对任意光滑映射$f:M\to N$与$\omega\in\form^k(N)$, 都有$f^*\d\omega=\d(f^*\omega)$.
\end{prop}
\begin{proof}
    仍然只需要考虑任意的一个坐标卡.
    设$\displaystyle\omega=\sum_{i_1,\cdots,i_k}A_{i_1,\cdots,i_k}\d{y^{i_1}}\wedge\cdots\wedge\d{y^{i_k}}$, 那么有
    \begin{align*}
        f^*\d\omega&=f^*\left(\sum_{i_1,\cdots,i_k}\d{A_{i_1,\cdots,i_k}}\wedge\d{y^{i_1}}\wedge\cdots\wedge\d{y^{i_k}}\right)\\
        &=\sum_{i_1,\cdots,i_k}\d{(A_{i_1,\cdots,i_k}\circ f)}\wedge\d{(y^{i_1}\circ f)}\wedge\cdots\wedge\d{(y^{i_k}\circ f)}\\
        &=\d\left(\sum_{i_1,\cdots,i_k}(A_{i_1,\cdots,i_k}\circ f)\d{(y^{i_1}\circ f)}\wedge\cdots\wedge\d{(y^{i_k}\circ f)}\right)\\
        &=\d\left(f^*\left(\sum_{i_1,\cdots,i_k}A_{i_1,\cdots,i_k}\d{y^{i_1}}\wedge\cdots\wedge\d{y^{i_k}}\right)\right)\\
        &=\d(f^*\omega)\qedhere
    \end{align*}
\end{proof}

最后我们讨论微分形式的一个简单应用: 我们可以使用微分形式判断流形是否可定向.

\begin{prop}\label{orientable and n-form}
    $n$维微分流形$M$可定向的充分必要条件是$M$上存在处处非零的$n$-形式.
\end{prop}

\begin{lem}\label{nowhere vanish}
    设$M$是连通的微分流形, $f\in C^\infty(M)$处处非零, 那么$f$一定恒正或恒负.
\end{lem}
\begin{proof}
    由于连续映射将连通集映为连通集~\Parencite[定理4.7]{Lee_IntroTopoMani}, 所以$f(M)\subset\mathbb{R}$一定是连通的.
    如果$f(M)$中同时出现了正数与负数, 那么按连通性, 一定有$0\in f(M)$, 与假设矛盾.
\end{proof}

\begin{proof}[{\bf 命题~\ref{orientable and n-form}~的证明}]
    {\itshape 必要性}. 我们取一组定向相同的图册$\{U_\alpha,\varphi_\alpha\}_{\alpha\in A}$, 并取一列从属于此图册的单位分解$\{f_k\}_{k\in\mathbb{N}}$.
    设$\supp{f_k}\subset U_{\alpha_k}$, $U_{\alpha_k}$上的余向量场为$\d{x^i_k}$, 那么取
    \[\omega_k|_p=\left\{\begin{array}{ll}
        f_k(p)\d{x^1_k}|_p\wedge\cdots\wedge\d{x^n_k}|_p, & p\in U_{\alpha_k} \\
        0, & \text{其他情况}
    \end{array}\right.\]
    并取
    \[\omega=\sum_{k\in\mathbb{N}}\omega_k\]
    由局部有限性知上式收敛.
    对任意的一点$p\in M$, 设包含$p$的支集 (经过重排后) 为$\supp{f_1},\cdots,\supp{f_r}$.
    那么在包含$\supp{f_1}$的坐标卡$(U_{\alpha_1},\varphi_{\alpha_1})$上有
    \begin{align*}
        \omega|_p&=\sum_{i=1}^r(\varphi_{\alpha_i}^{-1}\circ\varphi_{\alpha_1})^*(f_i\d{x^1_i}|_p\wedge\cdots\wedge\d{x^n_i}|_p)\\
        &=\sum_{i=1}^r\det{\d(\varphi_{\alpha_1}^{-1}\circ\varphi_{\alpha_i})}(f_i\circ\varphi_{\alpha_1}^{-1}\circ\varphi_{\alpha_i})\d{x^1_1}|_p\wedge\cdots\wedge\d{x^n_1}|_p
    \end{align*}
    由于每个$\det{\d(\varphi_{\alpha_1}^{-1}\circ\varphi_{\alpha_i})}$与$f_i\circ\varphi_{\alpha_1}^{-1}\circ\varphi_{\alpha_i}$都是恒正的, 所以$\omega|_p\neq 0$.
    又由于$p$的任意性, 可知$\omega$在$M$上恒非零.\\
    {\itshape 充分性}. 设$\omega$是一个在$M$上恒非零的$n$-形式, 我们按如下方式构造一个图册$\mscrU$:
    %对$M$的微分结构中的一个图册$(U_\alpha,\varphi_\alpha)$, $(U_\alpha,\varphi_\alpha)\in\mscrU$当且仅当存在恒正的$A_\alpha\in C^\infty(U_\alpha)$使得
    对任意$p\in M$, 取一个包含$p$的坐标卡$(U_p,\varphi_p)$, 必要时将$U_p$缩小为$p$所在的连通分支, 那么在$U_p$上$\omega$可以表示为
    \[\omega=A_p\d{x_p^1}\wedge\cdots\wedge\d{x_p^n}\]
    由引理~\ref{nowhere vanish}~可知, $A_p$恒正或恒负.
    如果恒负的话, 将$\varphi_p$复合一个对称变换可以使得$A_p$恒正.
    取$\mscrU=\{(U_p,\varphi_p)|\ p\in M\}$, 那么$\mscrU$是一个图册.
    我们证明$\mscrU$中的坐标卡可以决定一个定向.
    对任意交非空的$(U_p,\varphi_p),(U_q,\varphi_q)\in\mscrU$, 设在$U_p\cap U_q$上$\omega$分别表示为$A_p\d{x_p^1}\wedge\cdots\wedge\d{x_p^n}$与$A_q\d{x_q^1}\wedge\cdots\wedge\d{x_q^n}$, 那么一定有
    \[A_p\d{x_p^1}\wedge\cdots\wedge\d{x_p^n}=\det\d(\varphi_q^{-1}\circ\varphi_p)(A_q\circ\varphi_q^{-1}\circ\varphi_p)\d{x_p^1}\wedge\cdots\wedge\d{x_p^n}\]
    从而
    \[A_p=\det\d(\varphi_q^{-1}\circ\varphi_p)(A_q\circ\varphi_q^{-1}\circ\varphi_p)\]
    但$A_p,A_q$都是恒正的, 所以一定有$\det\d(\varphi_q^{-1}\circ\varphi_p)>0$, 从而$M$是可定向的.
\end{proof}

\section{Riemann流形}
第\ref{chapter_manifolds}章与第~\ref{section_tensorfield}, \ref{section_forms}~节的内容属于流形的\textit{微分拓扑}, 而我们在这一节将会引入我们在第~\ref{metric_first}~节提到的属于\textit{微分几何}的一个结构: \textit{Riemann度量}.

Riemann度量是欧氏空间中内积的推广, Riemann度量让我们能够在流形上定义长度, 角度, 面积等等一系列几何量, 并给出较之前几章多得多的几何信息.
同时, Riemann度量也使得\textit{等距变换群}有了意义, 从而我们进入了Erlangen纲领的轨道上, 正式开始了``几何学''的讨论.

\begin{defn}
    流形$M$上的一个\textbf{Riemann度量张量} (也简称为\textit{度量张量}或\textit{度量}) 是$M$上的一个对称, 正定的$2$阶共变张量场, 通常记为$\langle\quad,\quad\rangle$.
    一个流形与其上的一个度量$(M,\langle\quad,\quad\rangle)$被称为一个\textbf{Riemann流形}.
\end{defn}

\begin{rem}
    在许多采用经典Riemann几何语言的教材上 (如~\parencites{doCarmo_DiffForm}{Tu_DiffGeo}~等), Riemann流形有一个等价的定义:
    Riemann流形$(M,\langle\quad,\quad\rangle)$是一个流形$M$与在其每一点的切空间$T_pM$上光滑地指派的一个内积$\langle\quad,\quad\rangle_p$, 光滑意为对任意$X,Y\in\mathfrak{X}(M)$都有$\langle X,Y\rangle\in C^\infty(M)$.
\end{rem}
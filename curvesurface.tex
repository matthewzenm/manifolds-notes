\chapter{\texorpdfstring{$\mathbb{R}^3$}{R³}中的曲线与曲面}\label{chapter_curvesurface}
本章是前后那些``抽象而恐怖''的章节之间的一个休息, 我们来讨论一下古典曲线曲面论的内容.
我们在本章主要关心曲率这个局部的几何量:
对于曲线采用Frenet-Serret标架的方式定义曲率和挠率, 并证明曲线论基本定理;
对于曲面则采用形状算子的方式定义Gauss曲率, 但更深入的Gauss绝妙定理与曲面论基本定理则留待讨论联络与活动标架之后再证明.

\section{曲线的参数}
回忆定义~\ref{def of curve and hypersurface}~中对于曲线的定义, 我们要求曲线是一个$1$维的子流形.
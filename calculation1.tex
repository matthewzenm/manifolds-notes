\chapter{计算}
前面我们遇到的大量几何量实际上都是\textit{张量}.
我们打算在这一章提供一些必要的代数预备知识, 并介绍张量入门.
本章中所有的环都有单位元.

\section{代数知识}

\begin{defn}\label{def of left module}
    一个环$R$上的\textbf{左模}由一个Abel群$M$与$R$到$M$的自同态群的一个同态构成.
    换言之, 任意一个$r\in R$都诱导了一个$M\to M$的映射, 对$m\in M$, $m$在$r$诱导的映射下的像记为$rm$.
    我们要求这些映射满足如下公理:
    \begin{enumerate}
        \item 对$r\in R$, $m,n\in M$有$r(m+n)=rm+rn$;
        \item 对$r,s\in R$, $m\in M$有$(r+s)m=rm+sm$;
        \item 对$r,s\in R$, $m\in M$有$(rs)m=r(sm)$;
        \item 对任意$m\in M$有$1m=m$.
    \end{enumerate}
\end{defn}

\begin{defn}
    一个环$R$上的\textbf{右模}由一个Abel群$M$与$R$到$M$的自同态群的一个反同态构成.
    即将定义~\ref{def of left module}~中的公理3变为$(rs)m=s(rm)$.
\end{defn}

\begin{rem}
    如果$R$是交换的, 那么左模和右模是一样的, 称为\textbf{双侧模}或者直接简称为\textbf{模}.
    此外, 我们也可以把右模的像写成$mr$, 那么``数乘的结合公理''\footnote{打引号是因为我们只对向量空间说数乘, 模一般说作用.}可以写成很好看的形式$m(sr)=(ms)r$.
\end{rem}